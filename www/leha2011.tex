@Article{leha:2011,
  AUTHOR =	 {Leha, Andreas and Beissbarth, Tim and Jung, Klaus},
  TITLE =	 {Sequential Interim Analyses of Survival Data in DNA
                  Microarray Experiments},
  JOURNAL =	 {BMC Bioinformatics},
  VOLUME =	 {12},
  YEAR =	 {2011},
  NUMBER =	 {1},
  PAGES =	 {127},
  URL =		 {http://www.biomedcentral.com/1471-2105/12/127},
  DOI =		 {10.1186/1471-2105-12-127},
  PubMedID =	 {21527044},
  ISSN =	 {1471-2105},
  ABSTRACT =	 {BACKGROUND:Discovery of biomarkers that are
                  correlated with therapy response and thus with
                  survival is an important goal of medical research on
                  severe diseases, e.g. cancer. Frequently, microarray
                  studies are performed to identify genes of which the
                  expression levels in pretherapeutic tissue samples
                  are correlated to survival times of
                  patients. Typically, such a study can take several
                  years until the full planned sample size is
                  available. Therefore, interim analyses are
                  desirable, offering the possibility of stopping the
                  study earlier, or of performing additional
                  laboratory experiments to validate the role of the
                  detected genes. While many methods correcting the
                  multiple testing bias introduced by interim analyses
                  have been proposed for studies of one single
                  feature, there are still open questions about
                  interim analyses of multiple features, particularly
                  of high-dimensional microarray data, where the
                  number of features clearly exceeds the number of
                  samples. Therefore, we examine false discovery rates
                  and power rates in microarray experiments performed
                  during interim analyses of survival studies. In
                  addition, the early stopping based on interim
                  results of such studies is evaluated. As stop
                  criterion we employ the achieved average power rate,
                  i.e. the proportion of detected true positives, for
                  which a new estimator is derived and compared to
                  existing estimators.
                  RESULTS:In a simulation study,
                  pre-specified levels of the false discovery rate are
                  maintained in each interim analysis, where reduced
                  levels as used in classical group sequential designs
                  of one single feature are not necessary. Average
                  power rates increase with each interim analysis, and
                  many studies can be stopped prior to their planned
                  end when a certain pre-specified power rate is
                  achieved. The new estimator for the power rate
                  slightly deviates from the true power rate but is
                  comparable to other estimators.
                  CONCLUSIONS:Interim
                  analyses of microarray experiments can provide
                  evidence for early stopping of long-term survival
                  studies. The developed simulation framework, which
                  we also offer as a new R package 'SurvGenesInterim'
                  available at survgenesinter.R-Forge.R-Project.org,
                  can be used for sample size planning of the
                  evaluated study design.},
}